\chapter{Uvod}
Generativni modeli široka su klasa modela čiji je zajednički cilj reprezentacija neke visokodimenzionalne vjerojatnosne distribucije. Svoju primjenu su pronašli u brojnim područjima - sinteza slika visoke rezolucije iz niskorezolucijskih, prevođenje slike u sliku, generiranje teksta, sinteza realističnih slika, klasifikacija, učenje reprezentacije i slično.
Njihova primjena na slikama posebno je interesantno područje budući da je krajnjem korisniku dobar rezultat vrlo lako detektirati temeljem poznavanja strukture koju elementi na slici moraju poštivati. Primjerice, ljudi općenito vrlo lako prepoznaju psa na slici temeljem njihove apstraktne pretpostavke kako pas izgleda (četiri noge, karakterističan oblik njuške, krzno i slično). Međutim, generativni modeli moraju 1) naučiti apstraktnu strukturu i 2) koliko koji faktori te strukture mogu varirati (npr. pas može biti smeđe boje, ali rijetko jarko crvene). I to samo iz memorijske reprezentacije slika u računalu! Zato nije ni čudo da je ovaj izazov u isto vrijeme težak, kao i područje intenzivnog istraživanja.
Značajan napredak na ovom problemu jest formulacija klase modela pod nazivom \textit{generativne suparničke mreže}, koja se dogodila relativno nedavno - 2014. godine \citep{orig_paper}. Vrlo brzo su se pokazale izvrsnim izborom za različite varijante problema koji uključuju sintezu slika te su zaokupile interes brojnih istraživača.
U ovom radu, cilj je predstaviti teorijsku podlogu generativnih suparničkih mreža te predstaviti njihove gradivne elemente koji se uobičajeno koriste u praksi, kao i metode kojima ih vrednujemo. Nakon toga, predstavit ćemo implementaciju konkretne arhitekture i njezin uspjeh na referentnom skupu slika.