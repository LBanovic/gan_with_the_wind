\chapter{Uvod}
\todo{više zanimljivih primjera}
Generativne suparničke mreže su nova klasa modela, predložen 2014. godine \todo{referenca na originalni rad}, ali već su pokazale zavidne rezultate na mnogim problemima, pogotovo pri sintezi slika različitih vrsta. Primjerice, nedavno razvijene inačice \todo{ref stylegan} omogućuju generiranje slika visoke rezolucije koje je teško razlučiti od stvarnih slika, uz očuvanje određene razine kontrole nad apstraktnim značajkama objekta na slici, kao što su spol ili izraz lica u slučaju čovjeka. 

Nadalje, jedna od zanimljivijih primjena je i transfer stila između dvije domene \todo{cyclegan ref}. Uzmimo za primjer opus dvojice umjetnika poznatih po karakterističnom stilu, Claudea Moneta i Vincenta Van Gogha. Navedeni model sposoban je naučiti značajke njihova stila visoke razine, bez eksplicitnog uparivanja odgovarajućih slika, te ih primijeniti tako da referentnu sliku promijeni iz jednoga stila u drugi.

Možemo primijetiti da se radi o vrlo moćnom konceptu, primjenjivom na širok spektar problema. Cilj je ovoga rada predstaviti teoretsku podlogu funkcioniranja generativnih suparničkih mreža, zatim predstaviti elemente implementacije koji se standardno koriste u praksi. Nakon toga, bit će predstavljena konkretna implementacija modela te njezin uspjeh na referentnom skupu slika. Na kraju, bit će predstavljeni smjerovi u kojima istraživanje generativnih suparničkih mreža napreduje.

\todo{glatki prijelaz na definicije}
\todo{zašto generativni modeli, primjene}