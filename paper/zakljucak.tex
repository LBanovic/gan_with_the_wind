\chapter{Zaključak}
U ovom smo radu predstavili koncept generativnih suparničkih mreža u kontekstu generiranja slika. Između ostalog, pozabavili smo se njihovom osnovnom idejom, teorijskom podlogom te predstavili elemente od kojih se sastoje, zajedno s pregledom relevantnih radova. Napokon, prikazali smo primjenu koncepta na skupu slika koji se uobičajeno koristi u praksi te razjasnili odluke koje smo morali donijeti pri implementaciji.

Iz svega navedenoga, možemo zaključiti da su generativne suparničke mreže izrazito moćan model s vrlo dobrim rezultatima u generiranju slika. Međutim, brojna pitanja u ovom području su još uvijek neodgovorena.

Jedan od najvažnijih problema jest nestabilnost njihova treniranja, odnosno visoka osjetljivost na promjenu hiperparametara, kao i reproducibilnost rezultata. Osim toga, važan je izazov i mogućnost kontrole pri generiranju slika. Iako određeni napretci u tom području postoje, ideja ugradnje pretpostavki u generiranje slika je tek u svom početnom stadiju. I za kraj, neadekvatno odgovoreno pitanje je i problem evaluacije generiranih slika, odnosno prevencija prenaučenosti. 

Bilo kako bilo, možemo očekivati da će, uz stvaranje sve boljih i boljih skupova podataka kao i porast računalne moći, i ovi izazovi  biti razriješeni. 
